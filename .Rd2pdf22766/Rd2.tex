\documentclass[a4paper]{book}
\usepackage[times,hyper]{Rd}
\usepackage{makeidx}
\usepackage[utf8,latin1]{inputenc}
% \usepackage{graphicx} % @USE GRAPHICX@
\makeindex{}
\begin{document}
\chapter*{}
\begin{center}
{\textbf{\huge rGMAP}}
\par\bigskip{\large \today}
\end{center}
\begin{description}
\raggedright{}
\item[Type]\AsIs{Package}
\item[Title]\AsIs{Call hierarchical chromatin domains from HiC matrix by GMAP}
\item[Version]\AsIs{1.3}
\item[Date]\AsIs{06-12-2018}
\item[Author]\AsIs{Wenbao Yu}
\item[Maintainer]\AsIs{Wenbao Yu }\email{yuw1@email.chop.edu}\AsIs{}
\item[Description]\AsIs{Call hierarchical chromatin domains from HiC contact matrix by Gaussian Mixture model And Proportion test}
\item[BugReports]\AsIs{}\url{https://github.com/wbaopaul/rGMAP/issues}\AsIs{}
\item[License]\AsIs{GPL (>= 2)}
\item[LazyData]\AsIs{TRUE}
\item[Imports]\AsIs{data.table,
ggplot2,
mclust,
EMD,
caTools,
Matrix,
Rcpp (>= 0.12.5)}
\item[LinkingTo]\AsIs{Rcpp}
\item[RoxygenNote]\AsIs{6.0.1}
\item[Suggests]\AsIs{knitr,
rmarkdown}
\item[VignetteBuilder]\AsIs{knitr}
\end{description}
\Rdcontents{\R{} topics documented:}
\inputencoding{utf8}
\HeaderA{data\_simu}{generate simulated hic\_mat and true tads}{data.Rul.simu}
%
\begin{Description}\relax
generate simulated hic\_mat and true tads
\end{Description}
%
\begin{Usage}
\begin{verbatim}
data_simu(stype = "poisson-dist", nratio = 2.5, mu0 = 20, resl = 1)
\end{verbatim}
\end{Usage}
%
\begin{Arguments}
\begin{ldescription}
\item[\code{stype}] One of four types of simulated data in the manuscript:
poission-dist, poission-dist-hier, nb-dist, nb-dist-hier;
poission- or nb- indicates poission distribution or negative bionomial distribution
-hier indicated subtads are generated nestly

\item[\code{nratio}] The effect size between intra- and inter domain, larger means higher intra-tad contacts

\item[\code{mu0}] The mean parameter, default 20

\item[\code{resl}] Resolution, default set to 1
\end{ldescription}
\end{Arguments}
%
\begin{Value}
A list includes following elements:
\begin{ldescription}
\item[\code{hic\_mat}] n by n contact matrix
\item[\code{hierTads}] True heirarchical domains
\item[\code{tads\_true}] True TADs
\end{ldescription}
\end{Value}
\inputencoding{utf8}
\HeaderA{hic\_rao\_IMR90\_chr15}{Normalized Hi-C data for IMR90, chr15 with resolution 10kb.}{hic.Rul.rao.Rul.IMR90.Rul.chr15}
\keyword{datasets}{hic\_rao\_IMR90\_chr15}
%
\begin{Description}\relax
Normalized Hi-C data for IMR90, chr15 with resolution 10kb.
\end{Description}
%
\begin{Usage}
\begin{verbatim}
hic_rao_IMR90_chr15
\end{verbatim}
\end{Usage}
%
\begin{Format}
A data table with 3 variables:
\begin{description}

\item[n1] bin 1
\item[n2] bin 2
\item[count] normalized counts

\end{description}
\end{Format}
%
\begin{Source}\relax
Rao et al., Cell 2014, A 3D map of the human genome at kilobase resolution reveals principles of chromatin looping
\end{Source}
\inputencoding{utf8}
\HeaderA{plotdom}{visualize hierarchical domains}{plotdom}
%
\begin{Description}\relax
visualize hierarchical domains
\end{Description}
%
\begin{Usage}
\begin{verbatim}
plotdom(hic_dat, hiertads_gmap, start_bin, end_bin, cthr = 20, kb_resl = 10)
\end{verbatim}
\end{Usage}
%
\begin{Arguments}
\begin{ldescription}
\item[\code{hic\_dat}] hic contact matrix for a given chromosome with 3 columns
matrix or data.frame with columns: bin1, bin2, counts

\item[\code{hiertads\_gmap}] the hierarchical domains called by GMAP

\item[\code{start\_bin}] the start bin of the genome

\item[\code{end\_bin}] the end bin of the genome

\item[\code{cthr}] the upper bound count threshold for color, default 20

\item[\code{kb\_resl}] reslution of Hi-C data in kb
\end{ldescription}
\end{Arguments}
\inputencoding{utf8}
\HeaderA{rGMAP}{Detect hierarchical choromotin domains by GMAP}{rGMAP}
%
\begin{Description}\relax
Detect hierarchical choromotin domains by GMAP
\end{Description}
%
\begin{Usage}
\begin{verbatim}
rGMAP(hic_mat, resl = 10 * 10^3, logt = T, dom_order = 2,
  maxDistInBin = min(200, 2 * 10^6/resl), min_d = 25, max_d = 100,
  min_dp = 5, max_dp = 10, hthr = 0.95, t1thr = 0.5)
\end{verbatim}
\end{Usage}
%
\begin{Arguments}
\begin{ldescription}
\item[\code{hic\_mat}] Either a 3 columns Hi-C contact matrix for a given chromosome, with each row corrsponding to the start bin,
end bin and the contact number; or a n by n matrix, n is the number of bins for a given chromosom

\item[\code{resl}] The resolution (bin size), default 10kb

\item[\code{logt}] Do log-transformation or not, default TRUE

\item[\code{dom\_order}] Maximum level of hierarchical structures, default 2 (call TADs and subTADs)

\item[\code{maxDistInBin}] Only consider contact whose distance is not greater than maxDistInBIn bins,
default 200 bins (or 2Mb)

\item[\code{min\_d}] The minimum d (d: window size), default 25

\item[\code{max\_d}] The maximum d (d: window size), default 100

\item[\code{min\_dp}] The minmum dp (dp: lower bound of tad size), defalt 5

\item[\code{max\_dp}] The maximum dp (dp: lower bound of tad size), defalt 10.
min\_d, max\_d, min\_dp and max\_dp should be specified in number of bins

\item[\code{hthr}] The lower bound cutoff for posterior probability, default 0.95

\item[\code{t1thr}] Lower bound for t1 for calling TAD, default 0.5 quantile of test statistics
of TADs, 0.9 of subTADs
\end{ldescription}
\end{Arguments}
%
\begin{Value}
A list includes following elements:
\begin{ldescription}
\item[\code{tads}] A data frame with columns start, end indicates the start and end coordinates of each domain, respectively
\item[\code{hierTads}] A data frame with columns start, end, dom\_order, where dom\_order indicates the hierarchical status of a domain, 1 indicates tads, 2 indicates subtads, and so on
\item[\code{params}] A data frame gives the final parameters for calling TADs
\end{ldescription}
\end{Value}
%
\begin{Examples}
\begin{ExampleCode}
## On simulated data:
library(rGMAP)
simu_res = data_simu('poisson-dist-hier')
true_domains = simu_res$hierTads
simu_mat = simu_res$hic_mat
predicted_domains = rGMAP(simu_mat, resl = 1)$hierTads
true_domains
predicted_domains

## On an real data example
hic_rao_IMR90_chr15   # normalized Hi-C data for IMR90, chr15 with resolution 10kb
res = rGMAP(hic_rao_IMR90_chr15, resl = 10 * 1000, dom_order = 2)
names(res)
#quickly visualize some hierarchical domains
pp = plotdom(hic_rao_IMR90_chr15, res$hierTads, 6000, 7000, 30, 10)
pp$p2
\end{ExampleCode}
\end{Examples}
\printindex{}
\end{document}
